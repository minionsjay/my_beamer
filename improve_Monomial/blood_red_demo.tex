%!TeX encoding = UTF-8
%!TeX program = xelatex
\documentclass[notheorems,aspectratio=169]{beamer}
% aspectratio: 1610, 149, 54, 43(default), 32
\def\bd#1{\boldsymbol{#1}}
\usepackage{latexsym}
\usepackage{amsmath,amssymb}
\usepackage{mathtools}
\usepackage{color,xcolor}
\usepackage{graphicx}
\usepackage{algorithm}
\usepackage{algpseudocode} 
\usepackage{amsthm}
\usepackage{lmodern} % 解决 font warning
% \usepackage[UTF8]{ctex}
\usepackage{animate} % insert gif

\usepackage{lipsum} % To generate test text 
\usepackage{ulem} % 下划线,波浪线

\usepackage{listings} % display code on slides; don't forget [fragile] option after \begin{frame}

%\usepackage[ruled]{algorithm2e}
% tikx
\usepackage{framed}
\usepackage{tikz}
\usepackage{pgf}
\usetikzlibrary{calc,trees,positioning,arrows,chains,shapes.geometric,%
    decorations.pathreplacing,decorations.pathmorphing,shapes,%
    matrix,shapes.symbols}
\pgfmathsetseed{1} % To have predictable results
% Define a background layer, in which the parchment shape is drawn
\pgfdeclarelayer{background}
\pgfsetlayers{background,main}

% define styles for the normal border and the torn border
\tikzset{
  normal border/.style={orange!30!black!10, decorate, 
     decoration={random steps, segment length=2.5cm, amplitude=.7mm}},
  torn border/.style={orange!30!black!5, decorate, 
     decoration={random steps, segment length=.5cm, amplitude=1.7mm}}}

% Macro to draw the shape behind the text, when it fits completly in the
% page
\def\parchmentframe#1{
\tikz{
  \node[inner sep=2em] (A) {#1};  % Draw the text of the node
  \begin{pgfonlayer}{background}  % Draw the shape behind
  \fill[normal border] 
        (A.south east) -- (A.south west) -- 
        (A.north west) -- (A.north east) -- cycle;
  \end{pgfonlayer}}}

% Macro to draw the shape, when the text will continue in next page
\def\parchmentframetop#1{
\tikz{
  \node[inner sep=2em] (A) {#1};    % Draw the text of the node
  \begin{pgfonlayer}{background}    
  \fill[normal border]              % Draw the ``complete shape'' behind
        (A.south east) -- (A.south west) -- 
        (A.north west) -- (A.north east) -- cycle;
  \fill[torn border]                % Add the torn lower border
        ($(A.south east)-(0,.2)$) -- ($(A.south west)-(0,.2)$) -- 
        ($(A.south west)+(0,.2)$) -- ($(A.south east)+(0,.2)$) -- cycle;
  \end{pgfonlayer}}}

% Macro to draw the shape, when the text continues from previous page
\def\parchmentframebottom#1{
\tikz{
  \node[inner sep=2em] (A) {#1};   % Draw the text of the node
  \begin{pgfonlayer}{background}   
  \fill[normal border]             % Draw the ``complete shape'' behind
        (A.south east) -- (A.south west) -- 
        (A.north west) -- (A.north east) -- cycle;
  \fill[torn border]               % Add the torn upper border
        ($(A.north east)-(0,.2)$) -- ($(A.north west)-(0,.2)$) -- 
        ($(A.north west)+(0,.2)$) -- ($(A.north east)+(0,.2)$) -- cycle;
  \end{pgfonlayer}}}

% Macro to draw the shape, when both the text continues from previous page
% and it will continue in next page
\def\parchmentframemiddle#1{
\tikz{
  \node[inner sep=2em] (A) {#1};   % Draw the text of the node
  \begin{pgfonlayer}{background}   
  \fill[normal border]             % Draw the ``complete shape'' behind
        (A.south east) -- (A.south west) -- 
        (A.north west) -- (A.north east) -- cycle;
  \fill[torn border]               % Add the torn lower border
        ($(A.south east)-(0,.2)$) -- ($(A.south west)-(0,.2)$) -- 
        ($(A.south west)+(0,.2)$) -- ($(A.south east)+(0,.2)$) -- cycle;
  \fill[torn border]               % Add the torn upper border
        ($(A.north east)-(0,.2)$) -- ($(A.north west)-(0,.2)$) -- 
        ($(A.north west)+(0,.2)$) -- ($(A.north east)+(0,.2)$) -- cycle;
  \end{pgfonlayer}}}

% Define the environment which puts the frame
% In this case, the environment also accepts an argument with an optional
% title (which defaults to ``Example'', which is typeset in a box overlaid
% on the top border
\newenvironment{parchment}[1][Example]{%
  \def\FrameCommand{\parchmentframe}%
  \def\FirstFrameCommand{\parchmentframetop}%
  \def\LastFrameCommand{\parchmentframebottom}%
  \def\MidFrameCommand{\parchmentframemiddle}%
  \vskip\baselineskip
  \MakeFramed {\FrameRestore}
  \noindent\tikz\node[inner sep=1ex, draw=black!20,fill=white, 
          anchor=west, overlay] at (0em, 2em) {\sffamily#1};\par}%
{\endMakeFramed}

% ----------------------------------------------

\mode<presentation>{
    \usetheme{CambridgeUS}
    % Boadilla CambridgeUS
    % default Antibes Berlin Copenhagen
    % Madrid Montpelier Ilmenau Malmoe
    % Berkeley Singapore Warsaw
    \usecolortheme{beaver}
    % beetle, beaver, orchid, whale, dolphin
    \useoutertheme{infolines}
    % infolines miniframes shadow sidebar smoothbars smoothtree split tree
    \useinnertheme{circles}
    % circles, rectanges, rounded, inmargin
}
% 设置 block 颜色
\setbeamercolor{block title}{bg=red!30,fg=white}

\newcommand{\reditem}[1]{\setbeamercolor{item}{fg=red}\item #1}

% 缩放公式大小
\newcommand*{\Scale}[2][4]{\scalebox{#1}{\ensuremath{#2}}}

% 解决 font warning
\renewcommand\textbullet{\ensuremath{\bullet}}

% ---------------------------------------------------------------------
% flow chart
\tikzset{
    >=stealth',
    punktchain/.style={
        rectangle, 
        rounded corners, 
        % fill=black!10,
        draw=white, very thick,
        text width=6em,
        minimum height=2em, 
        text centered, 
        on chain
    },
    largepunktchain/.style={
        rectangle,
        rounded corners,
        draw=white, very thick,
        text width=10em,
        minimum height=2em,
        on chain
    },
    line/.style={draw, thick, <-},
    element/.style={
        tape,
        top color=white,
        bottom color=blue!50!black!60!,
        minimum width=6em,
        draw=blue!40!black!90, very thick,
        text width=6em, 
        minimum height=2em, 
        text centered, 
        on chain
    },
    every join/.style={->, thick,shorten >=1pt},
    decoration={brace},
    tuborg/.style={decorate},
    tubnode/.style={midway, right=2pt},
    font={\fontsize{10pt}{12}\selectfont},
}
% ---------------------------------------------------------------------

% code setting
\lstset{
    language=C++,
    basicstyle=\ttfamily\footnotesize,
    keywordstyle=\color{red},
    breaklines=true,
    xleftmargin=2em,
    numbers=left,
    numberstyle=\color[RGB]{222,155,81},
    frame=leftline,
    tabsize=4,
    breakatwhitespace=false,
    showspaces=false,               
    showstringspaces=false,
    showtabs=false,
    morekeywords={Str, Num, List},
}

% ---------------------------------------------------------------------

%% preamble
\title[Massive Superpoly Recovery with Nested Monomial Predictions]{Massive Superpoly Recovery with Nested Monomial Predictions}
% \subtitle{The subtitle}
\author{Kai Hu, Siwei Sun, Yosuke Todo, Meiqin Wang,  Qingju Wang}
\institute[HUST]{ASIACRYPT 2021}

% -------------------------------------------------------------
\usetheme{Boadilla}
\usefonttheme[onlymath]{serif}
\begin{document}

%% title frame
\begin{frame}
    \titlepage
\end{frame}

%% normal frame
\section{ Introduction}
%\subsection{}
\begin{frame}
    \frametitle{Introduction}
    %At EUROCRYPT 2009 Dinur and Shamir   proposed  cube attack against symmetric-key primitives with a secret key and a public input.   
    \begin{itemize}
        \item At EUROCRYPT 2009 Dinur and Shamir   proposed  cube attack
        \item At CRYPT 2017 the conventional bit-based division property was first introduced to probe the structure of the superpoly
        \item At EUROCRYPT 2020 Hao et al. proposed the three-subset division property without unknown subsets
    \end{itemize}
    \begin{block}{Contribution  }
        \begin{itemize}
         \item Propose a new framework with nested monomial predictions which scales well for massive superpoly recovery.\\
         \item With help of the Möbius transformation, we present a novel key-recovery technique based on superpolies involving all key bits exploiting the disjoint properties
        \end{itemize}
    \end{block}

\end{frame}


\begin{frame}
    \frametitle{Preliminaries}

    \begin{block}{Boolean Function }
        Let $f: \mathbb{F}^n_2 \rightarrow \mathbb{F}_2$ be a Boolean function whose algebraic normal form (ANF) is 
        $$f(\boldsymbol{x})=f(x_0,x_1,...,x_{n-1})=\bigoplus_{\boldsymbol{u}\in \mathbb{F}_2^n}a_{\boldsymbol{u}} \prod^{n-1}_{i=0}x_i^{u_i}$$
        where $a_\boldsymbol{u}\in \mathbb{F}_2$ and
        $$
        \boldsymbol{x}^{\boldsymbol{u}}=\pi_{\boldsymbol{u}}(\boldsymbol{x})=\prod^{n-1}_{i=0}x_i^{u_i}= \left\{\begin{matrix}x_i,if \quad u_i=1,\\1,if \quad u_i=0, \end{matrix}\right.    
        $$

    \end{block}
    \begin{example }{Example 1}
        Let $f(x_0,x_1) =x_0x_1 \oplus x_0 \oplus 1$ ,then we have  \centering $x_0x_1 \rightarrow f ,x_0 \rightarrow f,1 \rightarrow f ,x_1 \not \rightarrow f$ 
    \end{example }

\end{frame} 

\begin{frame}
    %\frametitle{An Example}

    
    \begin{block}{Vectorial Boolean Function }
        $\boldsymbol{f}:\mathbb{F}^m_2 \rightarrow \mathbb{F}^n_2 $  be a vectorial Boolean function with $\boldsymbol{y} =(y_0,y_1,...,y_{m-1})=\boldsymbol{f}(\boldsymbol{x})=(f_0(\boldsymbol{x}),...,f_{n-1}(\boldsymbol{x}))$ .
        For $\boldsymbol{x}\in \mathbb{F}^n_2 $,we use $\boldsymbol{y}^\boldsymbol{v}$ to denote the product of some coordinates of $\boldsymbol{y}$:
        $$ \boldsymbol{y}^v =\prod^{m-1}_{i=0}y_i^{v_i} =\prod ^{m-1}_{i=0}(f_i(\boldsymbol{x}))^{v_i}$$
    \end{block}
    

    
\end{frame}
\begin{frame}
    \frametitle{ Monomial Prediction}
    Let $\boldsymbol{f}:\mathbb{F}^{n_0}_2 \rightarrow \mathbb{F}^{n_r}_2$ be a composite vectorial Boolean function of a sequence of
    r smaller function $\boldsymbol{f}^i:\mathbb{F}^{n_i}_2 \rightarrow \mathbb{F}^{n_{i+1}}_2,0 \le i \le r-1$ as
    \begin{align}
        \boldsymbol{f}=\boldsymbol{f}^{(r-1)} \circ \boldsymbol{f}^{(r-2)} \circ \dots \circ \boldsymbol{f}^{(0)}
    \end{align}
    \begin{block}{Definition 1 (Monomial Trail)}
        Let $\bd{x}^{(i+1)}=\bd{f}^{(i)}(\bd{x}^{(i)})$ for $0\le i<r$.We call a sequence of monomails 
        $(\pi_{\boldsymbol{u}^{(0)}}(\boldsymbol{x}^{(0)}) , \pi_{\boldsymbol{u}^{(1)}}(\boldsymbol{x}^{(1)}) , \dots ,\pi_{\boldsymbol{u}^{(r)}}(\boldsymbol{x}^{(r)}))$
        an r-round monomial trail connecting $\pi_{\boldsymbol{u}^{(0)}}(\boldsymbol{x}^{(0)})$ and
        $\pi_{\boldsymbol{u}^{(r)}}(\boldsymbol{x}^{(r)})$ with respect to the composite function
        $\bd{f}=\bd{f}^{(r-1)}\circ \bd{f}^{(r-2)}\circ \dots \circ \bd{f}^{(0)}$ if
        $$
        \pi_{\boldsymbol{u}^{(0)}}(\boldsymbol{x}^{(0)}) \rightarrow \pi_{\boldsymbol{u}^{(1)}}(\boldsymbol{x}^{(1)}) \rightarrow \dots \rightarrow \pi_{\boldsymbol{u}^{(r)}}(\boldsymbol{x}^{(r)})
        $$
        \\
        If there is at least one monomial trail connecting $\pi_{\boldsymbol{u}^{(0)}}(\boldsymbol{x}^{(0)})$
        and $\pi_{\boldsymbol{u}^{(r)}}(\boldsymbol{x}^{(r)})$,we write 
        $\pi_{\boldsymbol{u}^{(0)}}(\boldsymbol{x}^{(0)})\leadsto \pi_{\boldsymbol{u}^{(r)}}(\boldsymbol{x}^{(r)})$.
        Otherwise,$\pi_{\boldsymbol{u}^{(0)}}(\boldsymbol{x}^{(0)})\not \leadsto \pi_{\boldsymbol{u}^{(r)}}(\boldsymbol{x}^{(r)})$
    \end{block}
\end{frame}

\begin{frame}
    \begin{block}{Example 2}
        Let $\boldsymbol{z}=(z_0,z_1)=\boldsymbol{f}^{(0)}(y_0,y_1)=(y_0y_1,y_0 \oplus y_1),\boldsymbol{y}=(y_0,y_1)=
        \boldsymbol{f}^{(0)}(x_0,x_1,x_2)=(x_0\oplus x_1 \oplus x_2 ,x_0x_1 \oplus x_0 \oplus x_2)$ and 
        $ \boldsymbol{f}=\boldsymbol{f}^{(0)} \cric \boldsymbol{f}^{(1)}$.
        \\


            
        $\boldsymbol{y}$
        $$
            (y_0,y_1)^{(0,0)}=1,(y_0,y_1)^{(1,0)}=y_0=\underline{x_0}\oplus x_1\oplus x_2,(y_0,y_1)^{(0,1)}=y_1=x_0x_1\oplus 
             \underline{x_0} \\ 
              \oplus x_2,\\
            (y_0,y_1)^{(1,1)}=y_0y_1=x_0x_1x_2 \oplus x_0x_1\oplus x_1x_2 \oplus \underline{x_0} \oplus x_2.
        $$
        \\Then $$ x_0 \rightarrow y_0,x_0\rightarrow y_1,x_0 \rightarrow y_0y_1$$
        \\Similarly $$ (z_0,z_1)^{(0,0)}=1,(z_0,z_1)^{(1,0)}=z_0=y_0y_1,(z_0,z_1)^{(0,1)}=z_1=y_0\oplus y_1,
         \\(z_0,z_1)^{(1,1)}=z_0z_1=0$$

         Then connecting $ x_0$ and monomials of z:$$ x_0 \rightarrow y_0 \rightarrow z_1,x_0 \rightarrow y_1 \rightarrow z_1,x_0 \rightarrow y_0y_1 \rightarrow z_0$$

    \end{block}
\end{frame}
\begin{frame}
    
    \begin{block}{Lemma 1}
        $\pi_{\boldsymbol{u}^{(0)}}(\boldsymbol{x}^{(0)})\leadsto \pi_{\boldsymbol{u}^{(r)}}(\boldsymbol{x}^{(r)})$.
        if $\pi_{\boldsymbol{u}^{(0)}}(\boldsymbol{x}^{(0)})\rightarrow \pi_{\boldsymbol{u}^{(r)}}(\boldsymbol{x}^{(r)})$,and thus
        $\pi_{\boldsymbol{u}^{(0)}}(\boldsymbol{x}^{(0)})\not \leadsto \pi_{\boldsymbol{u}^{(r)}}(\boldsymbol{x}^{(r)})$
        implies $\pi_{\boldsymbol{u}^{(0)}}(\boldsymbol{x}^{(0)})\not \rightarrow \pi_{\boldsymbol{u}^{(r)}}(\boldsymbol{x}^{(r)})$
    \end{block}
    Considering Example 2, although $x_0  \leadsto z_1$,we have $x_0 \not \rightarrow z_1$ since 
    $$
    z_1=y_0 \oplus y_1 =\underline{x_0} \oplus x_1\oplus x_2\oplus x_0x_1 \oplus \underline{x_0} \oplus x_2=x_0x_1\oplus x_1.
    $$
    \begin{block}{Definition 2 (Monomial Hull)}
        For $\bd{f}$ with a  specific composition sequence,
        the monomial hull of $\pi_{\boldsymbol{u}^{(0)}}(\boldsymbol{x}^{(0)})$ and $\pi_{\boldsymbol{u}^{(r)}}(\boldsymbol{x}^{(r)})$,
        denoted by $\pi_{\boldsymbol{u}^{(0)}}(\boldsymbol{x}^{(0)}) \bowtie \pi_{\boldsymbol{u}^{(r)}}(\boldsymbol{x}^{(r)})$,is the set of
        all monomial trails connecting them. The number of trails in the
        monomial hull is called the \textbf{size} of the hull and is denoted by
        $|\pi_{\boldsymbol{u}^{(0)}}(\boldsymbol{x}^{(0)}) \bowtie \pi_{\boldsymbol{u}^{(r)}}(\boldsymbol{x}^{(r)})|$.


    \end{block}
\end{frame}
\begin{frame}
    \begin{block}{Example 3}
        Consider Example 2, the monomial hull of $x_0$ and $z_1$ is the set 
        $$
        x_0 \bowtie z_1=\{x_0 \rightarrow y_0 \rightarrow z_1,x_0 \rightarrow y_1 \rightarrow z_1\}
        $$
        Thus  the size of $x_0 \bowtie z_1$ is 2.Furthermore, since $x_0 \not \leadsto z_0z_1$,
        $x_0 \bowtie z_0z_1 =\emptyset $ and $|x_0 \bowtie z_0z_1|=0$.
    \end{block}
    \begin{block}{Theorem 1}
        Let $\boldsymbol{f}=\boldsymbol{f}^{(r-1)} \circ \boldsymbol{f}^{(r-2)} \circ \dots \circ \boldsymbol{f}^{(0)}$ defined as
        above. $\pi_{\boldsymbol{u}^{(0)}}(\boldsymbol{x}^{(0)}) \rightarrow \pi_{\boldsymbol{u}^{(r)}}(\boldsymbol{x}^{(r)})$
        if and only if 
        $$
        |\pi_{\boldsymbol{u}^{(0)}}(\boldsymbol{x}^{(0)})\bowtie  \pi_{\boldsymbol{u}^{(r)}}(\boldsymbol{x}^{(r)}) \equiv 1 \pmod{2}.
        $$
    \end{block}
\end{frame}

\begin{frame}
    \frametitle{Cube Attack}
    For a cipher with a secret key $k \in  \mathbb{F}^m_2$ and a public input $x \in \mathbb{F}^n_2$, a Boolean function $f(x,k)$
        $$
        f(\boldsymbol{x},\boldsymbol{k})=p(\boldsymbol{x}[\hat{\boldsymbol{u}}],\boldsymbol{k}) \cdot \boldsymbol{x}^{\boldsymbol{u}}
        +q(\boldsymbol{x},\boldsymbol{k})
        $$
        Let $\mathbb{C} ={\boldsymbol{x} \in \mathbb{F}^n_2:\boldsymbol{x}} \preceq \boldsymbol{u}$  
        $$
        \bigoplus_{\boldsymbol{x} \in \mathbb{C}_u}f(\boldsymbol{x},\boldsymbol{k})=\bigoplus_{\boldsymbol{x}\in \mathbb{C}_u}(p \cdot \boldsymbol{x}^{\boldsymbol{u}}+q(\boldsymbol{x},\boldsymbol{k}))=p
        $$
        
        \\It is easy to check that the superpoly of $\mathbb{C}_u$ is just the coeffcient of $\boldsymbol{x}^{\boldsymbol{u}}$ in the
        parameterized Boolean function $f(\boldsymbol{x},\bd{k}) $
        $$
        p(\boldsymbol{x}[\hat{\boldsymbol{u}}],\boldsymbol{k})=Coe(f(\bd{x},\bd{k}),\bd{x}^{\bd{u}}).
        $$

\end{frame}

\section{ Superpoly Recovery with Nested Monomial Predictions}
\begin{frame}
    \frametitle{ The Nested Framework}

    Given a parameterized Boolean function which consists of a sequence of simple
vectorial Boolean functions as
    $$
    \boldsymbol{f}=\boldsymbol{f}^{(r-1)} \circ \boldsymbol{f}^{(r-2)} \circ \dots \circ \boldsymbol{f}^{(0)}
    $$
    
    $$
        f(\bd{x},\bd{k})=\bigoplus_{\bd{t} \in \mathbb{F}^n_2 , \pi_{\bd{t}}(r-r_0)(\bd{s}^{(r-r_0)})\in \mathbb{S}^{(r-r_0)}} 
        \pi _{\bd{t}}^{(r-r_0)}(\bd{s}^{(r-r_0)})
    $$
    where $\mathbb{S}^{(r-r_0)}=\{\pi _{\bd{t}}^{(r-r_0)}(\bd{s}^{(r-r_0)}):\pi _{\bd{t}}^{(r-r_0)}(\bd{s}^{(r-r_0)}) \rightarrow f\}$
    \begin{block}{Compute Coe$( \pi _{\bd{t}}^{(r-r_0)}(\bd{s}^{(r-r_0)}),\bd{x}^{\bd{u}})}
            $\bd{s}^{(r-r_0)}$        is the
output vector of a new composite vectorial Boolean function as

        $$ 
        \bd{s}^{(r-r_0)}=\bd{f}^{(r-r_0-1)}\circ \bd{f}^{(r-r_0-2)} \cric \dots \cric \bd{f}^{(0)}
        $$
        then $\pi _{\bd{t}}^{(r-r_0)}(\bd{s}^{(r-r_0)})$ is a polynomial of (\bd{x},\bd{k})
    \end{block}
\end{frame}


\begin{frame}
    %\frametitle{How can we improve the diagnosis}
    Hence we can construct the MILP model to enumerate all feasible trails representing
    $\bd{k}^{\bd{v}}\bd{x}^{\bd{u}}\leadsto \pi_{\bd{t}(r-r_0)}(\bd{s}^{(r-r_0)})$
    \\ set a time limit $\tau^{(r-r_0)}$ for thr MILP model.We use 
    $$
    \mathcal{M}.TimeLimit \leftarrow \tau^{(r-r_0)}
    $$ 
    For each element in $\mathbb{S}^{(r-r_0)}$,the model of enumerating the trails will end up with three different kinds of status
    \begin{enumerate}
        \item The model is solved and infeasible, then Coe$(\pi _{\bd{t}}^{(r-r_0)}(\bd{s}^{(r-r_0)}) )$
        \item The model is solved and feasible, and all the solutions has been enumerated,then 
        Coe$(\pi _{\bd{t}}^{(r-r_0)}(\bd{s}^{(r-r_0)}) )$ are obtained
        \item The model is not solved in the time limit $\tau^{(r-r_0)}$
    \end{enumerate}
    
        $$\mathbb{S}^{(r-r_0)}=\mathbb{S}_0^{(r-r_0)} \bigcup  \mathbb{S}_0^{(r-r_0)}_p \bigcup \mathbb{S}_0^{(r-r_0)}_u$$
        \begin{itemize}
            \item $\mathbb{S}_0^{(r-r_0)}_0$ is called a solved-0 set ,case 1
            \item $\mathbb{S}_0^{(r-r_0)}_p$ is called a solved-p set ,case 2
            \item $\mathbb{S}_0^{(r-r_0)}_u$ is called a undecided set ,case 3
        \end{itemize}
    
\end{frame}


\begin{frame}
    %\begin{tikzpicture}[remember picture,overlay]  
        %\node<1->[xshift=0cm,yshift=0cm] at (current page.center) {
            \includegraphics[scale=0.3]{1.png}
    %\end{tikzpicture}
\end{frame}

\begin{frame}
    \includegraphics[scale=0.3]{2.png}
    \\The solved-0 set is discarded naturally since the elements in it have no contribution to
    Coe$(f,\bd{x}^{\bd{u}} )$. \\
    For the solved-p set , 
    $$
    p^{(r-r_0)}=\bigoplus_{\pi _{\bd{t}}^{(r-r_0)}(\bd{s}^{(r-r_0)}) \in \mathbb{S}_p^{(r-r_0)} }
    Coe(\pi _{\bd{t}}^{(r-r_0)}(\bd{s}^{(r-r_0)}) )
    $$
    
\end{frame}
\subsection{Key-Recovery}
\begin{frame}
    \frametitle{Key-Recovery Attacks Exploiting Massive Superpolies}
    \begin{block}{offline}
        We have recovered the exact ANF of superpoly $p(\bd{f})$ for the cube term $\bd{x}^{\bd{u}}$ (the 
        corresponding cube is denoted by $\mathbb{C}_{\bd{u}}$).
    \end{block}
    \begin{block}{online}
        In the online phase, we first call the cipher oracle to encrypt all elements in the cube and get the value of
        the superpoly with time complexity $2^{wt(\bd{u})}$.\\
        Next,we try to obtain some information of the secret key from the equation:
        $$
        p(\bd{k})=\bigoplus _{\bd{x}\in \mathbb{C}_{\bd{u}}} f_{\bd{k}}(\bd{x}).
        $$
        
    \end{block}
\end{frame}
\begin{frame}
    It is well known that Möbius transformation is available for the conversion 
    between the ANF and the truth table of any Boolean function.
    \begin{algorithm}[H]
        \caption{Möbius transformation }
        \begin{algorithmic}[1]
            \Procedure {MöbiusTransformation}{$(a[i],0\le i\le 2^n):$}
            \For{$k=1$ to $n$}
            \For{$i=0$ to $2^{n-k}$}
            \For{$j=0$ to $2^{k-1}-1$}
            \State $a[2^ki+2^{k-1}+j]=a[2^ki+j]\oplus a[2^ki+2^{k-1}+j]$ 

        \Return{$a$}
        \end{algorithmic}
    \end{algorithm}
    It requires {\color{red}$n \times 2^{n-1}$} 1-bit XORs and {\color{red}$2^n-bit$} memory complexity.\\
    Considering the sparse property,the efficient algorithm, the Möbius transformation
    costs only  {\color{red}$n \times 2^{n-2}$} XORs for the superpolies we consider in this paper
        
\end{frame}
\begin{frame}
    \frametitle{Divide-and-Conquer Method Using the Disjoint Set}
    \begin{block}{Definition 1 (Disjoint set)}
        Given a superpoly $p(\bd{k})$ with n variables, if for $ 0\le i \not =j<n$,$k_i$ and $k_j$ are never multiplied mutually in all monomials
        of $p(\bd{k})$, then we say $k_i$ and $k_j$ are disjoint. If for a subset of variables $D\subseteq \{k_0,k_1,...,k_{n-1}\}$, 
        every pair of variables like $k_i,k_j \in D$ are all disjoint,we call D a disjoint set.
    \end{block}
    Search for a disjoint set of $p(\bd{k})$
        A matrix $M \in \mathbb{F}^n_2$ is called the disjoint matrix of $p(\bd{k})$,if $M[i][j]=0$ when $k_i,k_j$ are disjoint
        ,$M[i][j]=1$ otherwise.
        %Given the disjoint matrix, a locally-optimized disjoint set can be obtained by a greedy algorithm as follows,
        \begin{enumerate}
            \item sort the variables in $\{k_0,k_1,...,k_{n-1}\}$ in certain order,an increasing
            order according to the value $\sum_{0 \le j <n}M[i][j]$ for $k_i$.The sorted variables are
            denoted as$\{k_0^{\prime},k_1^{\prime},...,k_{n-1}^{\prime}\}$;
            \item initialize a set $D=\{k_0^{\prime}\}$
            \item for $1 \le i<n$,if $k_i^{\prime}$ is disjoint with all variables in D,put $k_i^{\prime}$ into D;
            otherwise,process the next variable
            \item after all the variables are processed,D is one of the disjoint sets.
        \end{enumerate}
    
\end{frame}
\begin{frame}
    
    \begin{block}{Key recovery attacks with single balanced superpoly}
        If the balanced superpoly $p(\bd{k})$ has a disjoint set $D$ with m variables and $J=\{k_0,k_1,...,k_{n-1}\}/D$,
        then $p(\bd{k})$ can be written as the form 
        $$
        p(k_0,k_1,...,k_{n-1})=\left(\bigoplus _{0\le i<m} k_i \cdot p_i(J)\right) \oplus p_m(J)
        $$

    \end{block}
    \textbf {Complexity}\\
    Every $p_i(J)$ involves at most n-m variables, Möbius
    transform to compute the truth tables of $p_0,p_1,...,p_m$ over all possible values of variables in J
    \\ For the linear equation, we can remove 1-bit key
    guessing efficiently after guessing m − 1 key bits additionally\\
    \centering {\color{red}$(m+1)\times (n-m) \times 2^{n-m-2}$} XORs to construct the truth tables\\
    \centering {\color{red}$2^{(m-1)}$} guesses for the values of any m −1 variables in the linear equations
\end{frame}
\begin{frame}
    \
\begin{block}{Key recovery attacks with multiple balanced superpolies}
    Suppose we have recovered N balanced superpolies $p^{(0)},p^{(1)},...,p^{(N-1)}$,we call $D$ their common disjoint set    
    \\The complexity of the case then consists of
\end{block}
\begin{enumerate}
    \item constructing the truth tables, which costs {\color{red}$N \times (m+1) \times (n-m) \times 2^{n-m-2}$} XORs;
    \item constructing the linear equations, which is {\color{red}$N\times 2^{n-m} \times (m+1)$} truth table lookuos;
    \item guessing the value of $m-N$(we always let $m>N$) cariables,then the remaining N variables can be determined by 
     solving a set of simple linear equations.This step costs {\color{red}$2^{n-m}\times x^{m-N}$} guesses.
\end{enumerate}

\end{frame}

\end{document}